A moi l'forban, que m'importe la gloire,\\
Les lois du monde, et qu'importe la mort ? \\
Sur l'océan j'ai planté ma victoire,\\
Et bois mon vin dans une coupe d'or. \\
Vivre d'orgie est ma seule espérance, \\
Le seul bonheur que j'aie pu conquérir. \\
C'est sur les flots qu'jai passé mon enfance, \\
C'est sur les flots qu'un forban doit mourir\\
~\\
\textbf{Vin qui pétille, femme gentille, \\
Sous tes baisers brûlants d'amour ; \\
Plaisirs, batailles, Vive la canaille ! \\
Je bois, je chante, et je tue tour à tour\\}
~\\
Peut-être qu'au mât d'une barque étrangère\\
Mon corps, un jour, servira d'étendard\\
Et tout mon sang rougira la galère\\
Aujourd'hui fête et demain le hasard. \\
Allons esclave, allons, debout mon brave,\\
Buvons la vie et le vin à grands pots ; \\
Aujourd'hui fête, et puis demain, peut-être \\
Ma tête ira s'engloutir dans les flots.\\
~\\
Peut-être qu'un jour, par un coup de fortune \\
Je capturerai l'or d'un beau gallion ;\\
Riche à pouvoir vous acheter la lune, \\
Je m'en irai vers d'autres horizons. \\
Là, respecté, comme un vrai gentilhomme, \\
Moi qui ne fus qu'un forban, qu'un bandit, \\
Je pourrai, comme le fils d'un roi, tout comme \\
Mourir, peut-être, dedans un grand lit.\\
~\\
